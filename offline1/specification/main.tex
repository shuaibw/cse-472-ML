\documentclass[11pt, a4paper]{article}

% Setting up the page geometry
\usepackage[margin=1in]{geometry}

% Including necessary packages for text formatting and math
\usepackage{amsmath, amssymb} % Math symbols and environments
\usepackage{mathtools} % Enhanced math typesetting
\usepackage{enumitem} % Customizable lists
\usepackage{booktabs} % Professional tables
\usepackage{titlesec} % Custom section formatting
\usepackage{parskip} % Paragraph spacing
\usepackage{hyperref} % Hyperlinks for references
\usepackage{url} % URL formatting

% Configuring hyperlinks
\hypersetup{
    colorlinks=true,
    linkcolor=blue,
    urlcolor=blue,
    citecolor=blue
}

% Setting up fonts (using standard Computer Modern for PDFLaTeX compatibility)
\usepackage{mathptmx} % Times-like font for math and text
\usepackage[T1]{fontenc} % Proper font encoding
\usepackage[utf8]{inputenc} % UTF-8 input encoding

% Customizing section titles
\titleformat{\section}{\large\bfseries}{\thesection}{1em}{}
\titleformat{\subsection}{\normalsize\bfseries}{\thesubsection}{1em}{}
\titleformat{\subsubsection}{\normalsize\itshape}{\thesubsubsection}{1em}{}

% Customizing list spacing
\setlist[itemize]{itemsep=0.5em, parsep=0.5em}
\setlist[enumerate]{itemsep=0.5em, parsep=0.5em}

% Document begins
\begin{document}

% Title and header
\begin{center}
    \textbf{\Large Linear Algebra for Machine Learning} \\[0.5em]
    \textbf{CSE 472 Assignment 1} \\[0.5em]
    \normalsize Courtesy: Md. Tareq Mahmood, Assistant Professor (on leave), CSE, BUET
\end{center}

\vspace{1em}

% References section
\section*{References}
\begin{enumerate}
    \item Chapter 2 (Linear Algebra) of the ``Deep Learning Book'' by Aaron Courville, Ian Goodfellow, and Yoshua Bengio. \url{deeplearningbook.org/contents/linear_algebra.html}
    \item ``Introduction to Linear Algebra for Applied Machine Learning with Python.'' 1 Aug. 2020, \url{pabloinsente.github.io/intro-linear-algebra}.
\end{enumerate}

% Link to files
Find the necessary files here: \textbf{CSE 472 Assignment 1 Files}

% Task 1
\section{Task 1: Matrix Transformation}
Go through and run the notebook ``matrix-transformations-and-eigen-decomposition'' to get an intuition about:
\begin{itemize}
    \item How can a matrix transform a vector?
    \item What do columns of matrices mean in terms of transformation?
    \item What does eigenvector mean?
\end{itemize}
(We recommend you also read the whole of Chapter 2 of the Deep Learning Book.)

Then,
\begin{itemize}
    \item Change the cell values of matrix $\mathbf{M}$ (there can be two such matrices; find one of them) so that it does the following transformation.
    \item Run the whole notebook again and submit.
\end{itemize}

% Task 2
\section{Task 2: Eigen Decomposition}

\subsection{SubTask 2A: Random Matrix (random\_eigen.py)}
\begin{itemize}
    \item Take the dimensions of matrix $n$ as input.
    \item Produce a random $n \times n$ invertible matrix $\mathbf{A}$. For the purpose of demonstrating, every cell of $\mathbf{A}$ will be an integer.
    \item Perform Eigen Decomposition using NumPy's library function.
    \item Reconstruct $\mathbf{A}$ from eigenvalues and eigenvectors (refer to Section 2.7).
    \item Check if the reconstruction worked properly. (\texttt{np.allclose} will come in handy.)
    \item You should be able to explain how your code ensures that the way you generated $\mathbf{A}$ ensures invertibility.
\end{itemize}

\subsection{SubTask 2B: Symmetric Matrix (symmetric\_eigen.py)}
\begin{itemize}
    \item Take the dimensions of matrix $n$ as input.
    \item Produce a random $n \times n$ invertible symmetric matrix $\mathbf{A}$. For the purpose of demonstrating, every cell of $\mathbf{A}$ will be an integer.
    \item Perform Eigen Decomposition using NumPy's library function.
    \item Reconstruct $\mathbf{A}$ from eigenvalues and eigenvectors (refer to Section 2.7).
    \item Check if the reconstruction worked properly. (\texttt{np.allclose} will come in handy.)
    \item Please be mindful of applying efficient methods (this will bear marks).
    \item You should be able to explain how your code ensures that the way you generated $\mathbf{A}$ ensures invertibility and symmetry.
\end{itemize}

% Task 3
\section{Task 3: Image Reconstruction using Singular Value Decomposition (image\_reconstruction.py)}
\begin{itemize}
    \item Take a photo of a book's cover within your vicinity. Let's assume it is named \texttt{image.jpg}.
    \item Use OpenCV or similar frameworks to read \texttt{image.jpg}. Transform it to grayscale using functions such as \texttt{cv2.cvtColor()}. If you wish, resize to lower dimensions ($\sim 500$) for faster computation.
    \item The grayscale image will be an $n \times m$ matrix $\mathbf{A}$.
    \item Perform Singular Value Decomposition using NumPy's library function.
    \item Given a matrix $\mathbf{A}$ and an integer $k$, write a function \texttt{low\_rank\_approximation(A, k)} that returns the $k$-rank approximation of $\mathbf{A}$.
    \item Now vary the value of $k$ from 1 to $\min(n, m)$ (take at least 10 such values in the interval). In each case, plot the resultant $k$-rank approximation as a grayscale image. Observe how the images vary with $k$. You can find a sample intended output in the shared folder.
    \item Find the lowest $k$ such that you can clearly read out the author's name from the image corresponding to the $k$-rank approximation.
\end{itemize}

% Marking Rubric
\section{Marking Rubric}
\begin{table}[h]
    \centering
    \begin{tabular}{ll}
        \toprule
        \textbf{Task} & \textbf{Percentage} \\
        \midrule
        Task 1 & 20\% \\
        Task 2A & 15\% \\
        Task 2B & 15\% \\
        Task 3 & 50\% \\
        \bottomrule
    \end{tabular}
\end{table}

Since most of the tasks you have to do here are basically invocations of library functions, it is expected that you understand the underlying concepts properly to get full marks.

% Submission
\section{Submission}
\begin{verbatim}
1805xyz
  |-- matrix-transformations-and-eigen-decomposition.ipynb
  |-- random_eigen.py
  |-- symmetric_eigen.py
  |-- image_reconstruction.py
  |-- image.jpg
\end{verbatim}
Zip the folder and rename it to \texttt{[Student\_ID].zip}.

\textbf{Deadline}: November 25, 2023, Saturday, 10 PM

\end{document}
